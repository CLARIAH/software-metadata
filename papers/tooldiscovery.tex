%!TEX encoding = UTF-8 Unicode
%!TEX TS-program = xelatex

\documentclass[a4paper,11pt]{article}
\usepackage{CLARIN2024}
% - - - - - - - IMPORTANT - - - - - - - -
% The next three lines allow XeLaTeX, graphics import and hyperlinks, set font and language
\usepackage{xltxtra,polyglossia,graphicx,hyperref}
\setmainfont[Mapping=tex-text]{Times}
\setdefaultlanguage{english}
% If for some reason the above three lines are not compatible with your LaTeX installation,
% comment out the above three instructions and uncomment the following four instead:
%\usepackage{times}
%\usepackage{url}
%\usepackage{latexsym}
%\usepackage{hyperref}
\usepackage[english]{babel}
\usepackage{csquotes}
\usepackage[
    backend=biber,
    style=apa,
    natbib=true,
    ]{biblatex}
\addbibresource{bibliography.bib} % add your own bibliography into the format provided in this file

%\setlength\titlebox{5cm}

% You can expand the titlebox if you need extra space
% to show all the authors. Please do not make the titlebox
% smaller than 5cm (the original size); we will check this
% in the camera-ready version and ask you to change it back.

%\usepackage{covington} % if needed, for linguistic examples

\title{CLARIAH Tool Discovery: an automated software metadata harvesting pipeline}

% - - - - - - - IMPORTANT - - - - - - -
% Leave the author information empty until your paper has been accepted 

% Uncomment the following line ONLY if you need two author rows
%\setlength\titlebox{80mm} 

\author{Maarten van Gompel \\
  KNAW Humanities Cluster \\
  Amsterdam, the Netherlands \\
  {\tt proycon@anaproy.nl} \\\And % if needed: this makes a second column
  Second Author \\
  Department (optional)\\
  University Name without city \\
  City, Country \\
 {\tt email@domain} \\
% \AND % if needed: this makes a second row
%  Third Author \\
%  Department (optional)\\
%  University of City, Country \\
%  {\tt email@domain} \\\And
%  Fourth Author \\
%  Department (optional)\\
%  University Name without city \\
%  City, Country \\
%  {\tt email@domain} \\
} 

\date{}

\begin{document}
\maketitle
\begin{abstract}
  We present the Tool Discovery pipeline, a core component of the Dutch CLARIAH infrastructure. This pipeline
  harvests software metadata from the source, converts it to a uniform representation based on schema.org and codemeta,
  and makes the resulting data available for further ingestion into user-facing portal systems. 
\end{abstract}

\section{Introduction} \label{intro}

% - - - - - - - IMPORTANT - - - - - - -
% The following footnote without marker is needed for the camera-ready
% version of the paper.
% Comment out the instructions (first text) and uncomment the 8 lines
% under "final paper" for your variant of English.
%
\blfootnote{
    %
    % for review submission
    %
    %\hspace{-0.65cm}  % space normally used by the marker
This work is licenced under a Creative Commons Attribution 4.0 International Licence. Licence details: {http://creativecommons.org/licenses/by/4.0/}
    % % final paper: en-uk version (to license, a licence)
    %
    % \hspace{-0.65cm}  % space normally used by the marker
    % This work is licensed under a Creative Commons
    % Attribution 4.0 International Licence.
    % Licence details:
    % {http://creativecommons.org/licenses/by/4.0/}
    %
    % % final paper: en-us version (to licence, a license)
    %
    % \hspace{-0.65cm}  % space normally used by the marker
    % This work is licenced under a Creative Commons
    % Attribution 4.0 International License.
    % License details:
    % {http://creativecommons.org/licenses/by/4.0/}
}

Software is indispensible in a lot of modern-day research, including in sectors
such as the Humanities and Social Sciences that may have traditionally been
less focused on information technology. It is also appreciated more and more as
valid research output, alongside more conventional output such as academic
publications, presentations, and datasets. Scholars often have a need for
research software to do their research efficiently.

For scholars it is therefore important to be able to find and identify
tools suitable for their research, we call this \emph{tool discovery}. We
define \emph{tool} here and throughout this paper to broadly refer to any kind
of software, regardless of the interface it offers and the audience it targets.
The scholar's requirement to find tools is expressed as the letter \textsc{F}
for \emph{Findable} in the ubiquitous acronym \textsc{FAIR} \footnote{Findable,
Accessible, Interoperable and Reusable} that has been popularised in recent
years. In order to find tools, researchers must have access to catalogues that
relay \emph{accurate} software metadata.

There is no shortage in existing initiatives in building such catalogues; many
research groups, projects or institutes have some kind of website featuring
their tools. Aggregation of software metadata from multiple such partners is
also not new, as CLARIN itself already does in the CLARIN Virtual Language
Observatory.

The system we describe in this paper is not an attempt to build another catalogue.
We developed a generic pipeline that harvests software metadata from the source, leveraging
various existing metadata formats, and converts it to a uniform linked open data representation.
This we then make available for further ingestion into catalogues. 

\section{Accurate and complete software metadata}

Unlike most digital data, software is uniquely characterised as a constantly
moving target. New releases are expected periodically in which bugs or security
vulnerabilities are fixed, or new features requested by users are added.
Updates also address the fact that software lives not in isolation, but in
connection to other software, its dependencies. Software regularly needs to
adapt to changes in its environment.

For software metadata to be accurate, it needs to describe a
particular version of the software, and be explicit about that version and in
what stage of development the software is. After all, if software is not
updated in a while, then that in itself is important metadata for an end-user
to have; the software may have been abandoned or become unmaintained. The user
would be wise to exercise caution in adopting such software, as there may be no
support, or if may not even install anymore.

A common pitfall we have observed in practice is that metadata is often
manually collected at some stage and published in a catalogue, but this is then
never or rarely updated or revised. In best case, the software has moved on and the
metadata covers a mere subset, in worst case, the software or the entire
catalogue is unmaintained and out of date.

Aside from accuracy, it is important that the metadata covers enough
information for the researcher to make an informed judgement on whether the
tool is suitable for them. Metadata must have a certain \emph{completeness} to
be useful.

\section{The source}

What we propose is a \emph{fully automated} pipeline where software metadata is kept
at the very source, which by definition is alongside the software source code, and
harvested from there. This has a number of important advantages and can be contrasted
with metadata that lives in a database of some intermediary:

\begin{itemize}
\item The software source code is often already accompanied by software metadata in an existing schema,
  because many programming language ecosystems already either require or recommend this.
  Consider for example \texttt{pyproject.toml} or \texttt{setup.py} for Python projects, \texttt{package.json} for javascript/npm/nodejs projects,
  \texttt{pom.xml} for Java/Maven and \texttt{Cargo.toml} for Rust. Multiple sources may be present.
\item Software source code is typically held in a version control system (usually git) and published in 
  forges such as Github, Gitlab, Bitbucket, Codeberg or Sourcehut. This solves versioning
  issues and ensures metadata can exactly describe the version alongside which it is stored.
\item The developers of the tool have full control and authorship over their metadata. There are no middlemen.
\item The forges were designed for open source collaboration, so mechanisms for any
  third party to amend or correct the metadata are also in place (e.g. via a pull/merge request or patches).
  So while developers retain full authorship, this does not mean curation is not possible.
\end{itemize}

We do not harvest any metadata from intermediaries as that would defeat our
philosphy. We do have one extra source of harvesting: In case the tool in
question is Software as a Service, i.e. a web-application, web-service, or
website, we harvest not only the software source code, but also a web endpoint
and attempt to automatically extract metadata from there. In the resulting
metadata, there will be an explicit link between the source code and any target
products that are \emph{instances} of that source code. These sources for
harvesting (URLs) are the only input that needs to be manually provided to our
system, we call this the \emph{source registry}. 

\section{Unified Vocabulary for software metadata}

The challenge we are facing is primarily one of mapping from multiple heterogeneous
sources of software metadata to a unified vocabulary. Fortunately, this is an area that
has been explored previously in the CodeMeta project. They established a
vocabulary for describing software source code, building on top of the
schema.org vocabulary and contributing their extensions back to them. Moreover,
the CodeMeta project defines mappings, called crosswalks, between their
vocabulary and many existing metadata schemes. 

Schema.org and codemeta are both linked open data (LOD) vocabularies\footnote{i.e.
building upon RDF and being retrievable over HTTP}, and codemeta is canonically
serialised to a JSON-LD file which makes it easily parsable for both machine
and human alike. This \texttt{codemeta.json} file can be kept under version
control alongside a tool's source code. 

We link to various other LOD vocabularies, such as
\emph{repostatus.org}\footnote{\url{https://repostatus.org}} (development
status), \emph{SPDX}\footnote{\url{https://spdx.dev}} (open source software
licenses), \emph{TaDiRaH}\footnote{\url{https://vocabs.dariah.eu/tadirah/}}
(research activities) and \emph{NWO Research Domains}. Moreover, we
formulated some of our own extensions on top of codemeta and schema.org, such
as Software
Types\footnote{\url{https://github.com/SoftwareUnderstanding/software_types}},
Software Input/Output
Data\footnote{\url{https://github.com/SoftwareUnderstanding/software-iodata}},
and Research Technology Readiness Levels. These are formulated as SKOS
vocabularies.

\section{Architecture}

The full architecture of our pipeline is illustrated schematically in
Figure~\ref{fig:architecture}.

\begin{figure}[h]
\includegraphics[width=14.0cm]{arch.png}
\caption{The CLARIAH Tool Discovery architecture}
\label{fig:architecture}
\end{figure}

Using the input from the source registry, our
\emph{harvester}\footnote{codemeta-harvester:
\url{https://github.com/proycon/codemeta-harvester}} fetches all the git
repositories and queries any service endpoints. It does so at regular intervals
(e.g. once a day), this ensures the metadata is always up to date. Subsequently, it
identifies the different kinds of metadata it can find there and and calls the
converter\footnote{Powered by codemetapy:
\url{https://github.com/proycon/codemetapy}} to turn and combine it into a
single codemeta representation. This produces one codemeta JSON-LD file per
input tool. All of these together are loaded in our \emph{tool store}. This is
implemented as a triple store and servers both as a backend to be queried programatically using SPARQL,
as well as a simple web frontend to be visited by human end-users as a catalogue
\footnote{codemeta-server
(\url{https://github.com/proycon/codemeta-server}) and codemeta2html
(\url{https://github.com/proycon/codemeta2html}). The results for CLARIAH are
publicly accessible at \url{https://tools.clariah.nl}}.

Our web frontend is not the endpoint; our aim is to propagate the metadata
we have collected to other existing portal/catalogue systems, such as the
CLARIN VLO, the CLARIN Switchboard, the SSHOC Marketplace, and CLARIAH's Ineo\footnote{\url{https://ineo.tools}}.

\section{Validation \& Curation}

Having an automated metadata harvesting pipeline may raise some concerns
regarding quality assurance. Data is automatically converted from heterogeneous
sources and immediately propagated to our tool store, this is not withotu
error. In absence of human curation, which is explicitly out of our intended
scope, we tackle this issue through an automatic validation mechanism.

The harvested codemeta metadata is held against a validation
schema\footnote{formulated in SHACL} that tests whether certain fields are
present (completeness), and whether the values are sensible (accuracy, it is
capable of detecting various discrepancies). The validation process outputs a
human-readable validation report which references a set of carefully formulated
software metadata requirements, this directly serves as documentation for
software producers so they can see what requirements they have not met. The
level of compliance to these requirements is expressed on a simple scale of 0
to 5, and visualised by stars and colours in our interface. This evaluation
score itself is part of the delivered metadata and something which both end
users as well as other systems can filter on. It may also serve as a kind of
'gamification' element to spur tool producers on to provide more accurate
metadata. 

For propagation to systems further downstream, we set a threshold of requiring
a metadata validation score of 3 or higher. These systems may also posit
whatever other criteria they want for inclusion into their systems, including
human validation and curation. As metadata is stored at the source, we strongly
recommend any curation efforts directly contributed upstream at the source,
through the established mechanisms in place by whatever forge (e.g. GitHub)
they are using to store their source code.

\section*{Acknowledgments}

Development of the tool discovery pipeline has been funded as part of the CLARIAH-PLUS project (NWO grant 184.034.023) 

\printbibliography

\end{document}
